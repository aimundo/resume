% Options for packages loaded elsewhere
\PassOptionsToPackage{unicode}{hyperref}
\PassOptionsToPackage{hyphens}{url}
%
\documentclass[
  11pt,
]{article}
\usepackage{lmodern}
\usepackage{amssymb,amsmath}
\usepackage{ifxetex,ifluatex}
\ifnum 0\ifxetex 1\fi\ifluatex 1\fi=0 % if pdftex
  \usepackage[T1]{fontenc}
  \usepackage[utf8]{inputenc}
  \usepackage{textcomp} % provide euro and other symbols
\else % if luatex or xetex
  \usepackage{unicode-math}
  \defaultfontfeatures{Scale=MatchLowercase}
  \defaultfontfeatures[\rmfamily]{Ligatures=TeX,Scale=1}
  \setmainfont[]{cochineal}
  \setmonofont[]{Fira Code}
\fi
% Use upquote if available, for straight quotes in verbatim environments
\IfFileExists{upquote.sty}{\usepackage{upquote}}{}
\IfFileExists{microtype.sty}{% use microtype if available
  \usepackage[]{microtype}
  \UseMicrotypeSet[protrusion]{basicmath} % disable protrusion for tt fonts
}{}
\makeatletter
\@ifundefined{KOMAClassName}{% if non-KOMA class
  \IfFileExists{parskip.sty}{%
    \usepackage{parskip}
  }{% else
    \setlength{\parindent}{0pt}
    \setlength{\parskip}{6pt plus 2pt minus 1pt}
    }
}{% if KOMA class
  \KOMAoptions{parskip=half}}
\makeatother
\usepackage{xcolor}
\IfFileExists{xurl.sty}{\usepackage{xurl}}{} % add URL line breaks if available
\urlstyle{same} % disable monospaced font for URLs
\usepackage[margin=0.5in]{geometry}
\setlength{\emergencystretch}{3em} % prevent overfull lines
\providecommand{\tightlist}{%
  \setlength{\itemsep}{0pt}\setlength{\parskip}{0pt}}
\setcounter{secnumdepth}{-\maxdimen} % remove section numbering

\ifluatex
  \usepackage{selnolig}  % disable illegal ligatures
\fi

\author{Ariel Mundo}
\date{14 March 2022}

% Jesus, okay, everything above this comment is default Pandoc LaTeX template. -----
% ----------------------------------------------------------------------------------
% I think I had assumed beamer and LaTex were somehow different templates.


\usepackage{kantlipsum}

\usepackage{abstract}
\renewcommand{\abstractname}{}    % clear the title
\renewcommand{\absnamepos}{empty} % originally center

\renewenvironment{abstract}
 {{%
    \setlength{\leftmargin}{0mm}
    \setlength{\rightmargin}{\leftmargin}%
  }%
  \relax}
 {\endlist}

\makeatletter
\def\@maketitle{%
  \newpage
%  \null
%  \vskip 2em%
%  \begin{center}%
  \let \footnote \thanks
      {\fontsize{18}{20}\selectfont\raggedright  \setlength{\parindent}{0pt} \@title \par}
    }
%\fi
\makeatother


 



%\author{\Large \vspace{0.05in} \newline\normalsize\emph{}  }


\date{}

\usepackage{titlesec}

% 
\titleformat*{\section}{\large\bfseries}
\titleformat*{\subsection}{\normalsize\itshape} % \small\uppercase
\titleformat*{\subsubsection}{\normalsize\itshape}
\titleformat*{\paragraph}{\normalsize\itshape}
\titleformat*{\subparagraph}{\normalsize\itshape}

% add some other packages ----------

% \usepackage{multicol}
% This should regulate where figures float
% See: https://tex.stackexchange.com/questions/2275/keeping-tables-figures-close-to-where-they-are-mentioned
\usepackage[section]{placeins}



\makeatletter
\@ifpackageloaded{hyperref}{}{%
\ifxetex
  \PassOptionsToPackage{hyphens}{url}\usepackage[setpagesize=false, % page size defined by xetex
              unicode=false, % unicode breaks when used with xetex
              xetex]{hyperref}
\else
  \PassOptionsToPackage{hyphens}{url}\usepackage[draft,unicode=true]{hyperref}
\fi
}

\@ifpackageloaded{color}{
    \PassOptionsToPackage{usenames,dvipsnames}{color}
}{%
    \usepackage[usenames,dvipsnames]{color}
}
\makeatother
\hypersetup{breaklinks=true,
            bookmarks=true,
            pdfauthor={ ()},
             pdfkeywords = {},  
            pdftitle={},
            colorlinks=true,
            citecolor=blue,
            urlcolor=blue,
            linkcolor=magenta,
            pdfborder={0 0 0}}
\urlstyle{same}  % don't use monospace font for urls

% Add an option for endnotes. -----



% This will better treat References as a section when using natbib
% https://tex.stackexchange.com/questions/49962/bibliography-title-fontsize-problem-with-bibtex-and-the-natbib-package



% set default figure placement to htbp
\makeatletter
\def\fps@figure{htbp}
\makeatother



\linespread{1.50}

\newtheorem{hypothesis}{Hypothesis}

\usepackage{fontawesome}

\newcommand{\blankline}{\quad\pagebreak[2]}
\usepackage{graphicx}

\begin{document}



\hfill
\begin{minipage}[t]{1\textwidth}
\raggedleft%
{\bfseries Ariel Mundo }\\[.35ex]
\emph{\small 1796 E. Parkshore Dr.~Apt. 4\\
Fayetteville, AR, US, 72703} \\[.35ex]
\faPhone \hspace{1 mm} \small{+1 479 800 8714} \\ 
\faEnvelopeO \hspace{1 mm} \small{\tt \href{mailto:aimundo@uark.edu}{\nolinkurl{aimundo@uark.edu}}} \\ 
\faGlobe \hspace{1 mm} \small{\href{http://aimundo.rbind.io}{\tt aimundo.rbind.io}}\\ 
\hspace{1cm} \\
 14 March 2022 \\ 
\end{minipage}

% \vspace*{1em} 

\vspace*{1em}

To whom it may concern:

\vspace*{1em}

 

% 14 March 2022
% 
% \vspace*{1em} 
% 

% % \setlength{\parindent}{16pt}
% \setlength{\parskip}{0pt}
% 
I am writing to express my interest in the Mathematics for Public Health
Postdoctoral Fellowship at the Fields Institute, particularly in the
Project {[}\textbf{project name here}{]}. I am a PhD Candidate in the
Department of Biomedical Engineering at the University of Arkansas,
where I investigate the longitudinal effects of chemotherapy in
colorectal cancer vascular development. My expected date of graduation
is the Spring of 2022.

Biological processes such as those involved cancer exhibit non-linear
trends over time. Over the course of my PhD studies, I have applied
generalized additive models to analyze biological processes involved in
cancer progression, thus being able to find specific time intervals
where significant molecular shifts occur and where novel therapies could
be investigated. For my postdoctoral training, I am interested in
applying and expanding my Biostatistics knowledge in a field different
from Oncology. Specifically, I want to focus on developing Statistical
methods that are appropriate to analyze time-dependent public health
data. My interest in this area has developed over the last two years,
where the necessity of better predictive strategies to address viral
transmission have become apparent in light of the ongoing Covid-19
pandemic. I believe that much can be done to develop strategies that can
better inform public health policies in a timely manner, and that
semi-parametric statistical methods can serve as a groundwork that can
be used to serve this purpose.

Therefore, I believe that my career interests align closely with the
goals of the Mathematics for Public Health project of the Fields
Institute; in addition, I consider that my unique combination of
expertise in the area of Disease Biology and Statistics makes me a
suitable candidate for the Public Health Postdoctoral Fellowship, as I
can work at the interface of both fields, thus providing a unique
perspective in the study of public health time-dependent data.

Thank you for taking the time to evaluate my application. I sincerely
hope to hear from you soon.

Sincerely,

Ariel Mundo

\end{document}
