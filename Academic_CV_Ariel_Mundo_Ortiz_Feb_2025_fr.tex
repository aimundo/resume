% CV ----
% This is just here so I know exactly what I'm looking at in Rstudio when messing with stuff.
\documentclass[10pt,]{article}
\usepackage[sc, osf]{mathpazo}
\usepackage{amssymb,amsmath}
\usepackage{ifxetex,ifluatex}
\usepackage{fixltx2e} % provides \textsubscript
\ifnum 0\ifxetex 1\fi\ifluatex 1\fi=0 % if pdftex
  \usepackage[T1]{fontenc}
  \usepackage[utf8]{inputenc}
\else % if luatex or xelatex
  \ifxetex
    \usepackage{mathspec}
  \else
    \usepackage{fontspec}
  \fi
  \defaultfontfeatures{Ligatures=TeX,Scale=MatchLowercase}
\fi
% use upquote if available, for straight quotes in verbatim environments
\IfFileExists{upquote.sty}{\usepackage{upquote}}{}
% use microtype if available
\IfFileExists{microtype.sty}{%
\usepackage{microtype}
\UseMicrotypeSet[protrusion]{basicmath} % disable protrusion for tt fonts
}{}
\usepackage[margin=0.5in]{geometry}




\setlength{\emergencystretch}{3em}  % prevent overfull lines
\providecommand{\tightlist}{%
  \setlength{\itemsep}{0pt}\setlength{\parskip}{0pt}}
\setcounter{secnumdepth}{0}
% Redefines (sub)paragraphs to behave more like sections
\ifx\paragraph\undefined\else
\let\oldparagraph\paragraph
\renewcommand{\paragraph}[1]{\oldparagraph{#1}\mbox{}}
\fi
\ifx\subparagraph\undefined\else
\let\oldsubparagraph\subparagraph
\renewcommand{\subparagraph}[1]{\oldsubparagraph{#1}\mbox{}}
\fi

% Now begins the stuff that I added.
% ----------------------------------

% Custom section fonts
\usepackage{sectsty}
\sectionfont{\rmfamily\mdseries\large\bf}
\subsectionfont{\rmfamily\mdseries\normalsize\scshape}


% Make lists without bullets
\renewenvironment{itemize}{
  \begin{list}{}{
    \setlength{\leftmargin}{1.5em}
  }
}{
  \end{list}
}


% Make parskips rather than indent with lists.
\usepackage{parskip}
% \usepackage{titlesec}
% \titlespacing\section{0pt}{12pt plus 4pt minus 2pt}{12pt plus 2pt minus 2pt}
% \titlespacing\subsection{0pt}{12pt plus 4pt minus 2pt}{12pt plus 2pt minus 2pt}

% Use fontawesome. Note: you'll need TeXLive 2015. Update.
\usepackage{fontawesome}

% Fancyhdr, as I tend to do with these personal documents.
\usepackage{fancyhdr,lastpage}
\pagestyle{fancy}
\renewcommand{\headrulewidth}{0.0pt}
\renewcommand{\footrulewidth}{0.0pt}
\lhead{}
\chead{}
\rhead{}
\lfoot{}
\cfoot{\scriptsize  Ariel Mundo Ortiz - Resume}
\rfoot{\scriptsize \thepage/{\hypersetup{linkcolor=black}\pageref{LastPage}}}

% Always load hyperref last.
\usepackage{hyperref}
\PassOptionsToPackage{usenames,dvipsnames}{color} % color is loaded by hyperref

\hypersetup{unicode=true,
            pdftitle={Ariel Mundo Ortiz:  Resume (Curriculum Vitae)},
            pdfauthor={Ariel Mundo Ortiz},
            pdfkeywords={R Markdown, academic CV, template},
            colorlinks=true,
            linkcolor=blue,
            citecolor=blue,
            urlcolor=blue,
            breaklinks=true, bookmarks=true}
\urlstyle{same}  % don't use monospace font for urls

% Make AP style (kinda) dates for the updated/today field

\usepackage{datetime}
\newdateformat{apstylekinda}{%
  \shortmonthname[\THEMONTH]. \THEDAY, \THEYEAR}

% \emph{Updated:} \apstylekinda\today
% ^ removed from that front bar.

\usepackage{orcidlink}
%\usepackage{academicons}
%\definecolor{orcidlogocol}{HTML}{A6CE39}
% \newcommand{\orcid}[1]{\href{https://orcid.org/#1}{\textcolor[HTML]{A6CE39}{\aiOrcid}}}

\begin{document}


\centerline{\huge \bf Ariel Mundo Ortiz}

\vspace{2 mm}

\hrule

\vspace{2 mm}

\moveleft.5\hoffset\centerline{Stagiaire postdoctoral}
\moveleft.5\hoffset\centerline{Centre de Recherche du CHUM}
\moveleft.5\hoffset\centerline{ \faEnvelopeO \hspace{1 mm} \href{mailto:}{\tt \href{mailto:aimundo.ortiz@gmail.com}{\nolinkurl{aimundo.ortiz@gmail.com}}} \hspace{1 mm}  \faPhone \hspace{1 mm}  +1
438 875
9203  \hspace{1 mm}       \faGlobe \hspace{1 mm} \href{http://aimundo.rbind.io}{\tt aimundo.rbind.io}    | \emph{Updated:} \apstylekinda\today}



\vspace{2 mm}

\hrule



\section{Formation}\label{formation}

\emph{Université de l'Arkansas}, Doctorat en génie biomédical
\hfill 2017-2022

\emph{Université Rafael Landivar (Guatemala)}, Licence en génie chimique
(avec distinction) \hfill 2005-2009

\section{Expérience de recherche}\label{expuxe9rience-de-recherche}

\emph{Stagiaire postdoctoral, Centre de Recherche du CHUM} (Montréal,
QC) \hfill janvier 2024 - aujourd'hui

Au sein du laboratoire NG du CRCHUM, je travaille sur des projets à
l'intersection de la bioinformatique, des signes vitaux et de la
médecine personnalisée. Je suis responsable du développement de chaînes
de traitement des données, des analyses statistiques, ainsi que de la
rédaction et de la révision d'articles scientifiques. Mes recherches
postdoctorales ont été publiées dans \emph{Practical Laboratory
Medicine} et \emph{The Journals of Gerontology: Series A}.

\emph{Stagiaire postdoctoral, Département de médecine sociale et
préventive et au Centre de recherche en santé publique, Université de
Montréal} (Montréal, QC) \hfill Juillet 2022 - Décembre 2023

Au Laboratoire Nasri de l'École de santé publique de Montréal (ESPUM),
j'ai mené des recherches en modélisation des maladies infectieuses,
incluant des analyses spatiales. Mes tâches comprenaient la rédaction,
la révision et l'édition d'articles scientifiques, la recherche
documentaire, la collaboration avec différents groupes de recherche, la
contribution à des projets de recherche subventionnés du laboratoire et
le développement de logiciels pour le nettoyage et l'analyse de données.
Ma première année de bourse postdoctorale a été financée par une bourse
CRM-MfPH. Mes travaux de recherche postdoctoraux ont été publiés dans
les revues \emph{Vaccine} et \emph{One Health}.

\emph{Étudiant au doctorat} (Université de l'Arkansas, Fayetteville, AR)
\hfill août 2017 - mai 2022

Élaboration de questions de recherche sur la réponse tumorale à la
chimiothérapie. Collecte de données et programmation pour les analyses
statistiques. Rédaction d'une demande de subvention en collaboration
avec le directeur de thèse. Publication des articles de recherche
doctorale dans \emph{Statistics in Medicine}, \emph{Neoplasia} et
\emph{Journal of Biomedical Optics}.

\section{Autres expériences
pertinentes}\label{autres-expuxe9riences-pertinentes}

\textbf{Université Rafael Landivar}

\emph{Professeur adjoint} \hfill 2016-2017

\begin{itemize}
\tightlist
\item
  Professeur de chimie au Département des sciences environnementales et
  agricoles. Responsabilités~: élaboration de cours, encadrement de
  travaux pratiques, mentorat d'étudiants, mise à jour des manuels de
  laboratoire pour réduire les coûts et l'impact écologique, et
  communication avec le corps professoral.
\end{itemize}

Professeur associé \hfill 2013-2017

\begin{itemize}
\tightlist
\item
  Enseignement de la chimie générale aux départements d'ingénierie,
  d'environnement et d'agriculture, et des sciences de la santé. Mes
  fonctions comprenaient l'élaboration de cours magistraux, la
  supervision de travaux pratiques, l'encadrement d'étudiants, la mise à
  jour des manuels de laboratoire afin de réduire les coûts et l'impact
  écologique, ainsi que la communication avec le corps professoral.
\end{itemize}

Lacteos Balcanicos Glad

Ingénieur adjoint d'usine \hfill 2012

\begin{itemize}
\tightlist
\item
  Responsable de la production de différents produits latiers. Mes
  fonctions comprenaient la supervision du personnel, la gestion des
  stocks, la planification de la production et le contrôle qualité.
\end{itemize}

\section{Publications}\label{publications}

\subsection{Articles de revue}\label{articles-de-revue}

\textbf{Mundo Ortiz, Ariel I.}, Emond Jean-Philippe, Cheung Vincent
Weng-Jy, Saeed Sahar, Desmarais Philippe, Larivière François, Hétu
Pierre-Olivier \& Nguyen Quoc Dinh. ``The impact of pathological
fluctuations versus biological variation on the interpretation of
laboratory values''. \emph{Practical Laboratory Medicine} (November
2025). \url{https://doi.org/10.1016/j.plabm.2025.e00511}.

Cheung Vincent Weng-Jy, Libotte Michaël, Nguyen Patrick Viet-Quoc, Minh
Vu Thien-Tuong, Émond Jean-Philippe, \textbf{Mundo Ortiz Ariel I.},
Desmarais Philippe,Nguyen Quoc Dinh. ``Preliminary feasibility and
development of a heart rate-based mobility and activity scale for
hospitalized older adults''. \emph{The Journals of Gerontology: Series
A} (November 2025) \url{https://doi.org/10.1093/gerona/glaf212}

El-Mousawi Fatima, \textbf{Mundo Ortiz Ariel I.}, Berkat Rawda, Nasri
Bouchra. ``The Impact of Flood Adaptation Measures on Affected
Population's Mental Health: A mixed method Scoping Review''.
\emph{Disaster Medicine and Public Health Preparedness} (2024).
\url{https://doi.org/10.1101/2023.04.27.23289166}

\textbf{Mundo Ortiz Ariel I.}, Nasri Bouchra. ``Socio-demographic
determinants of COVID-19 vaccine uptake in Ontario: Exploring
differences across the Health Region model''. Vaccine (2024).
\url{https://doi.org/10.1016/j.vaccine.2024.02.045}.

Molla Jeta, Sekkak Idriss, \textbf{Mundo Ortiz Ariel I.}, Moyles Iain,
Nasri Bouchra. ``Mathematical modeling of mpox: a scoping review''. One
Health (2023). \url{https://doi.org/10.1016/j.onehlt.2023.100540}

\textbf{Mundo Ortiz Ariel I.}, Muhammad Abdussaboor, Balza Kerlin,
Nelson Christopher E., Muldoon Timothy J. ``Longitudinal examination of
perfusion and angiogenesis markers in primary colorectal tumors shows
distinct signatures for metronomic and maximum-tolerated dose
strategies''. Neoplasia (2022).
\url{https://doi.org/10.1016/j.neo.2022.100825}

\textbf{Mundo Ortiz Ariel I.}, Tipton John R., Muldoon Timothy J.
``Generalized additive models to analyze non-linear trend sin biomedical
longitudinal data using R: Beyond repeated measures ANOVA and Linear
Mixed Models.'' Statistics in Medicine (2022).
\url{https://doi.org/10.1002/sim.9505}

\textbf{Mundo Ortiz Ariel I.}, Greening Gage, Fahr Michael J., Hale
Lawrence N., Bullard Elizabeth, Rajaram Narasimhan, and Muldoon Timothy
J. ``Diffuse reflectance spectroscopy to monitor murine colorectal tumor
progression and therapeutic response.'' Journal of Biomedical Optics
(2020). \url{https://doi.org/10.1117/1.JBO.25.3.035002}

Greening Gage, \textbf{Mundo Ortiz Ariel I.}, Rajaram Narasimhan,
Muldoon Timothy J. ``Sampling depth of a diffuse re-flectance
spectroscopy probe forin-vivophysiological quantification of murine
subcutaneous tumor allografts''.Journal of Biomedical Optics (2018).
\url{https://doi.org/10.1117/1.JBO.23.8.085006}

\subsection{Présentations orales}\label{pruxe9sentations-orales}

\textbf{Mundo Ortiz, Ariel I.} ``Reproducible papers in the life
sciences using R''. CANSSI Statistical Software Conference. November
2022. Recording: \url{https://www.youtube.com/watch?v=4yRAR9fS3pg}

\textbf{Mundo Ortiz, Ariel I.}. ``Statistics and Reproducibility in
Biomedical Research''. 2022 Toronto Workshop on Reproducibility.
Toronto, ON. February 2022. Recording:
\url{https://www.youtube.com/watch?v=Fvkp20X5xwA}

\textbf{Mundo Ortiz, Ariel I.}, Muldoon, Timothy J. ``Longitudinal
optical and molecular quantification provides insight into the effect of
different dosing strategies in colorectal cancer''. 2022 Biophotonics
Congress: Biomedical Optics, Fort Lauderdale, FL, USA, April 2022.

\textbf{Mundo Ortiz, Ariel I.} ``Statistics and Reproducibility in
Biomedical Research: Why we need both''. Toronto Workshop on
Reproducibility, University of Toronto, February 2022. Recording
available \href{https://www.youtube.com/watch?v=Fvkp20X5xwA}{here}.

\textbf{Mundo Ortiz, Ariel I.} ``Using generalized additive models for
biomedical longitudinal data. \emph{When linear models don't work}''.
RMedicine 2021 Conference. Recording: \url{https://tinyurl.com/39epnrp6}
Repository (slides and data):
\url{https://aimundo.rbind.io/talks/gams-biomedical/}

\textbf{Mundo Ortiz, Ariel I.}, Abdussaboor Muhammad, and Timothy J.
Muldoon. ``Optical and molecular longitudinal tracking of primary
colorectal murine tumors shows differences in the angiogenic response to
maximum-tolerated and metronomic approaches.'' In Label-free Biomedical
Imaging and Sensing (LBIS) 2021, vol.~11655, p.~116551C.
\emph{International Society for Optics and Photonics}, 2021.
\url{https://doi.org/10.1117/12.2576906}

\textbf{Mundo Ortiz, Ariel I.}, Elizabeth Bullard, Kyle P. Quinn, and
Timothy J. Muldoon. ``Optical spectroscopic and imaging biomarkers of
ulcerative colitis disease progression and remission (Conference
Presentation).'' In Multiscale Imaging and Spectroscopy, vol.~11216,
p.~1121605. \emph{International Society for Optics and Photonics}, 2020.
\url{https://doi.org/10.1117/12.2543369}

\textbf{Mundo Ortiz, Ariel I.}, Gage J. Greening, and Timothy Muldoon.
``Characterization of a multimodal endoscopically deployable veterinary
spectroscopy and imaging probe to determine therapeutic response in a
murine orthotopic tumor model.'' In Label-free Biomedical Imaging and
Sensing (LBIS) 2019, vol.~10890, p.~108901L. \emph{International Society
for Optics and Photonics}, 2019.

\section{Prix et distinctions}\label{prix-et-distinctions}

\emph{Fields Institute : Festival de mathématiques pour la santé
publique (MfPHFest)} \hfill Octobre 2022

Financement obtenu pour participer au congrès MfPHFest et présenter des
travaux postdoctoraux en statistique (1~500~\$ CA).

\emph{Centre de recherches mathématiques (CRM) : Bourse postdoctorale
MfPH-CRM} \hfill Juillet 2022

Bourse postdoctorale obtenue pour un an afin d'analyser des données en
santé publique et en comportement (50~000~\$ CA).

\emph{Boursier émergent en génie biomédical} \hfill Juin 2022

Sélectionné par le comité du programme de bourses en génie biomédical de
l'Université Washington à Saint-Louis pour bénéficier d'un mentorat sur
les parcours et la réussite professionnelle dans le milieu
universitaire. La bourse comprenait la prise en charge des frais de
voyage, de participation au congrès et une rémunération.

\emph{Boursier du programme PAAD} (Professional Awareness, Advancement,
and Development) \hfill 2020-2021

J'ai bénéficié d'un financement et participé au programme PADD afin de
compléter ma formation doctorale en communication persuasive,
commercialisation et science des données (1~500 USD).

\emph{Bourse internationale OMNI} \hfill 2020

Attribuée en tant que chercheur participant à la mission du Centre OMNI
de Fayetteville (500 USD).

\emph{Bourse Fulbright pour le développement du corps professoral}
\hfill 2017-2019

Cette bourse a couvert l'intégralité des frais de mon doctorat pendant
les deux premières années. Seules deux bourses ont été attribuées pour
cette période dans tout le pays. (78~000 USD)

\section{Subventions}\label{subventions}

\emph{Instituts de recherche en santé du Canada (IRSC)} \hfill 2023

Cochercheur associé au projet de subvention «~Modélisation
épidémiologique de l'impact comportemental sur les stratégies
d'atténuation des mpox~», dirigé par le professeur Nasri de l'Université
de Montréal.

Participation à la rédaction de la demande de subvention, à la revue de
la littérature, aux échanges avec les membres de l'équipe et à la
révision de la version finale.

Financement obtenu~: 412~000~\$ CA.

La proposition s'est classée parmi les 50~\% meilleures.

Le dossier de la subvention est disponible dans la base de données des
décisions de financement IRSC
\href{https://webapps.cihr-irsc.gc.ca/decisions/p/project_details.html?applId=481271&lang=en}{(lien)}.

\emph{Concours de subventions d'amorçage de l'Institut des biosciences
de l'Arkansas} \hfill 2021

Auteur principal d'une proposition co-soumise avec le Dr Timothy Muldoon
visant à étudier l'expression génique et les marqueurs optiques dans un
modèle murin de cancer colorectal.

Financement obtenu~: 30~000~USD.

La proposition s'est classée parmi les deux meilleures de tous les
projets de recherche individuels de ce cycle.

\section{Langages}\label{langages}

\begin{itemize}
\item
  Espagnol : Langue maternelle
\item
  Anglais : Maîtrise professionnelle complète
\item
  Français : Maîtrise professionnelle complète
\end{itemize}

\end{document}
