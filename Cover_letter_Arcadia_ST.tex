% Options for packages loaded elsewhere
\PassOptionsToPackage{unicode}{hyperref}
\PassOptionsToPackage{hyphens}{url}
%
\documentclass[
  11pt,
]{article}
\usepackage{lmodern}
\usepackage{amssymb,amsmath}
\usepackage{ifxetex,ifluatex}
\ifnum 0\ifxetex 1\fi\ifluatex 1\fi=0 % if pdftex
  \usepackage[T1]{fontenc}
  \usepackage[utf8]{inputenc}
  \usepackage{textcomp} % provide euro and other symbols
\else % if luatex or xetex
  \usepackage{unicode-math}
  \defaultfontfeatures{Scale=MatchLowercase}
  \defaultfontfeatures[\rmfamily]{Ligatures=TeX,Scale=1}
  \setmainfont[]{cochineal}
  \setmonofont[]{Fira Code}
\fi
% Use upquote if available, for straight quotes in verbatim environments
\IfFileExists{upquote.sty}{\usepackage{upquote}}{}
\IfFileExists{microtype.sty}{% use microtype if available
  \usepackage[]{microtype}
  \UseMicrotypeSet[protrusion]{basicmath} % disable protrusion for tt fonts
}{}
\makeatletter
\@ifundefined{KOMAClassName}{% if non-KOMA class
  \IfFileExists{parskip.sty}{%
    \usepackage{parskip}
  }{% else
    \setlength{\parindent}{0pt}
    \setlength{\parskip}{6pt plus 2pt minus 1pt}
    }
}{% if KOMA class
  \KOMAoptions{parskip=half}}
\makeatother
\usepackage{xcolor}
\IfFileExists{xurl.sty}{\usepackage{xurl}}{} % add URL line breaks if available
\urlstyle{same} % disable monospaced font for URLs
\usepackage[margin=0.5in]{geometry}
\setlength{\emergencystretch}{3em} % prevent overfull lines
\providecommand{\tightlist}{%
  \setlength{\itemsep}{0pt}\setlength{\parskip}{0pt}}
\setcounter{secnumdepth}{-\maxdimen} % remove section numbering

\ifluatex
  \usepackage{selnolig}  % disable illegal ligatures
\fi

\author{Ariel Mundo}
\date{13 March 2022}

% Jesus, okay, everything above this comment is default Pandoc LaTeX template. -----
% ----------------------------------------------------------------------------------
% I think I had assumed beamer and LaTex were somehow different templates.


\usepackage{kantlipsum}

\usepackage{abstract}
\renewcommand{\abstractname}{}    % clear the title
\renewcommand{\absnamepos}{empty} % originally center

\renewenvironment{abstract}
 {{%
    \setlength{\leftmargin}{0mm}
    \setlength{\rightmargin}{\leftmargin}%
  }%
  \relax}
 {\endlist}

\makeatletter
\def\@maketitle{%
  \newpage
%  \null
%  \vskip 2em%
%  \begin{center}%
  \let \footnote \thanks
      {\fontsize{18}{20}\selectfont\raggedright  \setlength{\parindent}{0pt} \@title \par}
    }
%\fi
\makeatother


 



%\author{\Large \vspace{0.05in} \newline\normalsize\emph{}  }


\date{}

\usepackage{titlesec}

% 
\titleformat*{\section}{\large\bfseries}
\titleformat*{\subsection}{\normalsize\itshape} % \small\uppercase
\titleformat*{\subsubsection}{\normalsize\itshape}
\titleformat*{\paragraph}{\normalsize\itshape}
\titleformat*{\subparagraph}{\normalsize\itshape}

% add some other packages ----------

% \usepackage{multicol}
% This should regulate where figures float
% See: https://tex.stackexchange.com/questions/2275/keeping-tables-figures-close-to-where-they-are-mentioned
\usepackage[section]{placeins}



\makeatletter
\@ifpackageloaded{hyperref}{}{%
\ifxetex
  \PassOptionsToPackage{hyphens}{url}\usepackage[setpagesize=false, % page size defined by xetex
              unicode=false, % unicode breaks when used with xetex
              xetex]{hyperref}
\else
  \PassOptionsToPackage{hyphens}{url}\usepackage[draft,unicode=true]{hyperref}
\fi
}

\@ifpackageloaded{color}{
    \PassOptionsToPackage{usenames,dvipsnames}{color}
}{%
    \usepackage[usenames,dvipsnames]{color}
}
\makeatother
\hypersetup{breaklinks=true,
            bookmarks=true,
            pdfauthor={ ()},
             pdfkeywords = {},  
            pdftitle={},
            colorlinks=true,
            citecolor=blue,
            urlcolor=blue,
            linkcolor=magenta,
            pdfborder={0 0 0}}
\urlstyle{same}  % don't use monospace font for urls

% Add an option for endnotes. -----



% This will better treat References as a section when using natbib
% https://tex.stackexchange.com/questions/49962/bibliography-title-fontsize-problem-with-bibtex-and-the-natbib-package



% set default figure placement to htbp
\makeatletter
\def\fps@figure{htbp}
\makeatother



\linespread{1.00}

\newtheorem{hypothesis}{Hypothesis}

\usepackage{fontawesome}

\newcommand{\blankline}{\quad\pagebreak[2]}
\usepackage{graphicx}

\begin{document}



\hfill
\begin{minipage}[t]{1\textwidth}
\raggedleft%
{\bfseries Ariel Mundo }\\[.35ex]
\emph{\small 1796 E. Parkshore Dr.~Apt. 4\\
Fayetteville, AR, 72703} \\[.35ex]
\faPhone \hspace{1 mm} \small{+1 479 800 8714} \\ 
\faEnvelopeO \hspace{1 mm} \small{\tt \href{mailto:aimundo@uark.edu}{\nolinkurl{aimundo@uark.edu}}} \\ 
\faGlobe \hspace{1 mm} \small{\href{http://aimundo.rbind.io}{\tt aimundo.rbind.io}}\\ 
\hspace{1cm} \\
 13 March 2022 \\ 
\end{minipage}

% \vspace*{1em} 

\vspace*{1em}

Dear Dr.~Avasthi and Dr.~Chou:

\vspace*{1em}

 

% 13 March 2022
% 
% \vspace*{1em} 
% 

% % \setlength{\parindent}{16pt}
% \setlength{\parskip}{0pt}
% 
I am writing to express my interest in the Scientist (Discovery Teams)
position at Arcadia Science. My PhD work has been focused on gaining a
better understanding of the metabolic changes caused by colon cancer
using a primary murine model. However, it has become clear to me that
the way we study any disease (cancer, diabetes, obesity, etc.) in
preclinical research is limited by two fundamental reasons: 1) we use
animal or cellular models that approximate the behavior of a disease in
humans, but that have many inherent limitations, and 2) most of the
time, we only take ``snapshots'' of biological processes.

The first limitation has repeatedly caused a lack of correspondence in
the clinical setting from many promising results seen in traditional
model organisms. Despite this, those same models are used again and
again without producing timely and impactful discoveries that can have a
direct effect on the lives of millions of people. The second limitation
is even more baffling to me: why are we so accustomed to see things only
``before'' and ``after''? Diseases are progressive! And yet, because we
routinely neglect to consider \emph{time} in our research questions we
lack fundamental metabolic and cellular information with enough temporal
resolution that enables us to predict not only \emph{how} things change
but also \emph{when} they change.

Until we acknowledge the need for longitudinal information in all areas
of science, we are condemned to see not only a lack of replicability in
results, but a growing disorganized array of information that will not
help us give definite answers to many biological problems (e.g., the
Warburg effect in cancer).

During my PhD I have started to tackle the limitations outlined above by
doing longitudinal studies to examine the metabolic changes caused by
cancer over time using optics and molecular biology. I have also spent a
significant amount of time learning and incorporating into my work
statistical methods that go beyond a mere ``p-value'' to assess
significance.

However, I recognize that my efforts so far are a minuscule part of what
is required to bring \emph{change}. My next career goal is to produce
discoveries that help address the limitations described above by using
information from novel model organisms, where metabolic questions can be
addressed from a different perspective and longitudinal data can be
generated and shared with the broad scientific community. In the long
run, I am also interested in laying the theoretical and practical
foundations on \emph{how} to address these questions systematically for
any biological process in an interdisciplinary manner where novel
approaches across Statistics, Biology, Bioinformatics, and Open Science
can be leveraged in order to help change what we consider ``paradigms''
in research.

Because my work expands multiple fields, I believe my unique scientific
superpower is perseverance, and I think it would be best used by
designing experiments where a multidisciplinary approach is required.

I believe my career goals align with the mission of Arcadia Science, and
I would be delighted to form part of a team that has Open Science and
innovative ideas at its core.

Thank you for taking the time to evaluate my application. I sincerely
hope to hear from you soon.

Best regards,

Ariel Mundo

\end{document}
